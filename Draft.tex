\documentclass[12pt]{scrartcl}
\usepackage[hidelinks]{hyperref}
\usepackage[utf8]{inputenc}
\usepackage[square]{natbib}
\usepackage{graphicx}
\usepackage{minted}
\usepackage{subcaption}

\title{Parallel detection of movement flock patterns in large trajectory databases}
\author{_}

\begin{document}

\maketitle
 
\section{Introduction}
Technology advances in last past decades have triggered an explosion in the capture of spatio-temporal data.  The increase popularity of GPS devices and smart-phones together with the emerging of new disciplines such as the Internet of Things and Satellite/UAS high-resolution imagery have made possible to collect vast amounts of data with a spatial and temporal component attached to them.

Together with this, the interest to extract valuable information from such large databases has also appeared.  Spatio-temporal queries about most popular places or frequent events are still useful, but more complex patterns are recently shown an increase of interest.  In particular, those what describe group behaviour through significant periods of time.

Applications of these kind of patterns are diverse and useful in different scenarios.  Collective behaviour can be used to support decisions in transport integrated systems and urban planning.  In Ecology, they have been used to understand the trends and patterns during animal migrations.  Even in software development, patterns in the ocular movement of users are been used to improve graphic user interfaces. 


\end{document}